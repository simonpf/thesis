
The previous chapter discussed atmospheric radiative transfer and the observable
signals that result from the presence hydrometeors in the atmosphere. Technical
limitations make that observation frequencies for hydrometeors cannot be chosen
freely, but typically constitute a trade-off between feasibility, cost and
scientific value. This chapter provides an overview of the available observations.
Based on this it will be discussed how these observations can be related to
physical properties of the atmosphere using the theory of inverse problems.

\section{Sensor types}

In addition to \texit{passive} sensors, which sense reflected solar radiation or
thermal radiation emitted from the Earth or its atmosphere, there exist sensors
that themselves emit radiation and measure the amount of it that is scattered
back to the sensor. By measuring the time of travel between emission and
reflection, these \textit{active} sensors measure the distance from the detector
in addition to the intensity of the reflected radiation this allows them to
profile the atmosphere along the line of sight, which leads to much higher
vertical resolutions than what can be achieved with atmospheric sounding.

While active sensors can provide unrivaled vertical resolution, their swath
width and thus horizontal coverage is typically limited. An additional
disadvantage of active sensors is that the strength of the scattering back to
the sensor is only indirectly related to the mass of the observed precipitation,
which causes uncertainties in the retrievals.

Passive sensors allow much larger swath sizes to be realized and thus provide
much higher spatial coverage. Furthermore, these sensors require less energy and
are cheaper than their passive counterparts. There are therefore more of them
available and they typically also provide better spectral coverage.

\section{Satellite platforms}

In addition to the sensor type, the satellite platform upon which the sensor is
placed plays an important role in determining the characteristics of the
resulting observations. Visible and infrared sensors is that they can produce
spatially highly resolved images of the Earth and its atmosphere. This allows
them to be deployed employed on geostationary satellite platforms. Geostationary
satellite orbit the Earth at an altitude of around $\SI{35\ 000}{\kilo \meter}$
at the same angular velocity as the Earth, which allows them to hover over the
same position of the Earth surface. In this way they can provide continuous
observations of large parts of the Earth surface that lie below the sensor.

Microwave observations require very large antennas to achieve high resolutions.
Since the antenna sizes that can be realized in space are limited, microwave
sensors are currently restricted to low earth orbits with altitudes between 500
and $\SI{1000}{\kilo \meter}$. Their field of view is variable and typically has
an extent of 1000 to $\SI{2000}{\kilo \meter}$ across the ground track of the
satellite. Depending on the exact size of the field of view, the time between
consecutive overpasses for a fixed position on Earth may be as high as
$\SI{12}{\hour}.


\section{Observation frequencies}

At visible and infrared wavelengths the scattering and absorption by clouds are
so strong that these observations are sensitive only to cloud tops. While this
helps to determine the altitude of the clouds as well as the properties of the
particles in those upper parts, these observations contain only indirect
information on the lower parts of the cloud or the occurrence of precipitation.

In the microwave domain, the interaction between radiation and hydrometeors are
much weaker. At frequencies below around $\SI{50}{\giga \hertz}$ the only signal
observed is emission from rain drops. This, however, requires a relatively dark
background and therefore typically only visible above water surfaces. At
frequencies between about $50$ and $\SI{200}{\giga \hertz}$ scattering from large
snow flakes causes a cold scattering signal in the warmer upwelling radiation.
Observations at these long wavelengths can thus provide information only on the
largest particles in the cloud, which are typically so large that they are
precipitating in the form of rain, snow, graupel or hail.


\section{From observations to measurements}
\label{sec:remote_sensing:retrieval_problem}

So far, the discussion of how satellite observations can inform us about the
state of the atmosphere was only qualitative. The question that remains to be
answered is how these satellite observations can be used to derive
\texit{quantitative} estimates of physical properties of the atmosphere. What
makes this difficult is that this problem typically does not allow a unique
solution. The reason for this is that the information from the satellite
observations is insufficient to fully determine the state in the atmosphere and
that these observations are affected by measurement errors.

Such problems can be addressed with the mathematical framework of
\textit{inverse problem theory}. Here the underlying assumption is that we are
observing a system, whose state is described by the vector $\mathbf{x} \in
\mathrm{R}^n$. The observations $\mathbf{y} \in \mathrm{R}^M$ are generated
through physical processes that can be described through a forward model
function $f: \mathrm{R}^n \rightarrow \mathrm{R}^m$ and may be affected by a
random error $\mathbf{\epsilon} \in \mathrm{R}^M$:
\begin{align}\label{eq:inverse_problem}
  \mathbf{y} &= f(\mathbf{x}) + \mathbf{\epsilon}
\end{align}

Inferring hydrometeor properties from satellites observations thus amounts to
finding the vector $\mathbf{x}$ that satisfies Eq.~\ref{eq:inverse_problem}. The
problem is that an exact solution to this problem does not exist. Because of the
random noise $\mathbf{\epsilon}$, the equation is never strictly satisfied.
Furthermore, observations $\mathbf{y}$ of clouds are often ambiguous in the
sense that a physically different hydrometeor configuration, for example a cloud
with more and smaller particles, produces similar observations to another one.

Although an exact solution to the problem is not possible, using Bayesian
statistics we can still make use of the observations in $\mathbf{y}$. In the
Bayesian framework, both the observations $\mathbf{y}$ and the atmospheric state
vector $\mathbf{x}$ are assumed to be random variables. Furthermore it is
assumed that $\mathbf{x}$ is distributed according to a known \textit{a priori}
distribution $p(\mathbf{x})$. Instead of a single state $\mathbf{x}$, the
solution of the problem then becomes the posterior distribution
$p(\mathbf{x} | \mathbf{y})$ of the atmospheric state. According to Bayes theorem,
this solution is given by:
\begin{align}\label{eq:bayes}
  p(\mathbf{x} | \mathbf{y}) &= \frac{p(\mathbf{y}|\mathbf{x}) p(\mathbf{x})}{p(\mathbf{y})}
\end{aling}

Equation \label{eq:bayes} can be solved in two different ways:
\begin{itemize}
  \time In the case that we have access to matching pairs $(\mathbf{x}, \matbf{y})$ of
  observations and atmospheric states, we try to infer the conditional probability
  $p(\mathbf{x} | \mathbf{y}$ directly from data. This the approach that is the focus of
  this thesis.
  \item Alternatively, if we can construct a computational model of the probability
  $p(\mathbf{y} | \mathbf{x})$, the right-hand side of Eq.~\ref{eq:bayes} can be used to
  calculate or approximate the posterior distribution.
\end{itemize}

Both of these approaches are commonly used in remote sensing. The disadvantage
of the former is that it requires a dataset with observations and corresponding
atmospheric states, which are not always simple to come by. The difficulty with
the latter approach is typically the calculation of the probability
$p(\mathbf{y}|\mathbf{x})$, which requires solving the radiative transfer
equation to simulate the observations.
