
The interactions of hydrometeors with electromagnetic radiation, which were
discussed in the previous chapter, provide the signal that can be measured
by earth observing satellites. This chapter now turns to the practical
aspects of the remote sensing of hydrometeors, namely how these signals
can be obtained and how they can be related into physical estimates.

\section{Satellite observations}



\subsection{Synergies}

\section{Satellite platforms}

\section{Satellite observations}


\section{Observation frequencies}

At visible and infrared wavelengths the scattering and absorption by clouds are
so strong that these observations are sensitive only to cloud tops. While this
helps to determine the altitude of the clouds as well as the properties of the
particles in those upper parts, these observations contain only indirect
information on the lower parts of the cloud or the occurrence of precipitation.

In the microwave domain, the interaction between radiation and hydrometeors are
much weaker. At frequencies below around $\SI{50}{\giga \hertz}$ the only signal
observed is emission from rain drops. This, however, requires a relatively dark
background and therefore typically only visible above water surfaces. At
frequencies between about $50$ and $\SI{200}{\giga \hertz}$ scattering from large
snow flakes causes a cold scattering signal in the warmer upwelling radiation.
Observations at these long wavelengths can thus provide information only on the
largest particles in the cloud, which are typically so large that they are
precipitating in the form of rain, snow, graupel or hail.


