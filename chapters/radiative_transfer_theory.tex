
This chapter now turns to the physical principles that enable satellites to
sense clouds and precipitation from space. The remote sensing of any atmospheric
quantity is based on its interaction with electromagnetic radiation. The
measured radiation can be used to infer physical properties of atmospheric
quantities by inverting the interaction between radiation and the atmosphere.
Doing so, requires a quantitative model of the propagation of radiation through
the atmosphere. Such a model is provided by the physical theory of radiative
transfer.

Since radiative transfer theory is fundamental to atmospheric remote sensing
(even, as we will see later on, it is not used directly
in the retrieval), this section provides an introduction to radiative transfer
in the atmosphere. The focus is put on the interaction of radiation with
hydrometeors. This presentation is mostly based on the more comprehensive texts
by Mishchenko et al. (2002), Thomas and Stamnes (2002), and Wallace and Hobbs
(2006).


\section{Radiative transfer theory}

Radiative transfer theory describes the propagation of electromagnetic radiation
through a medium. It is derived from electromagnetic field theory and provides
simplified model the transfer of energy by a perfectly monochromatic parallel
beam of light. Electromagnetic sensors measure the mean total energy that is
transferred by the electromagnetic radiation as well the correlations between
the components of the beam's electric field, which is referred to as the beam's
\textit{polarization state}. The radiation measured by any sensor at position $\bm{r}$ pointing in direction $\bm{-n}$ is fully described by the four dimensional stokes vector
\begin{align}
  \bm{I}(\bm{r}, \bm{n}) &= \left [ \begin{array}{c}
    I \\
    Q \\
    U \\
    V \\
    \end{array} \right ]
\end{align}

The components $I, Q, U$ and $V$ are called the stokes parameters and have the
unit of monochromatic energy flux. Since the stokes vector fully describes the
state of a monochromatic beam of radiation to the extent that it can be
measured, any observation from an electromagnetic sensor can be derived from the
stokes vector at the position of the sensor. The main challenge of atmospheric
radiative transfer is thus to model the evolution of $\bm{I}$ as it propagates
through the atmosphere.

\subsection{Interactions with matter}

Radiative transfer theory distinguishes three fundamental processes through which
matter interacts which radiation. These are  (1) the emission of electromagnetic radiation, (2) its absorption and (3) scattering.

\subsubsection{Emission}

At temperatures above absolute zero, all matter emits radiation through the
process of thermal emission. Thermal emission occurs when matter transitions
from a quantum mechanical state of higher energy to one of lower energy which
causes the excess energy to be emitted in the form of radiation. The amount of
radiation of a given frequency $\nu$ emitted by a body depends on its
temperature and material. It is typically modeled using a material-dependent
emissivity vector $\vec{e}$, which relates the emission of the material to that
of an ideal black body, i. e. a material that absorbs all incoming radiation:
 \begin{align}
   \label{eq:emissivity}
   \vec{I} &= (\vec{e} \cdot ds) B(T, \nu)
 \end{align}
 Here $B(T, \nu)$ is the radiation emitted by a black body at temperature $T$
 and frequency $\nu$, which is described by Planck's law
 \begin{align}
   B(T, \nu) &= \frac{2 \nu^2}{c^2}\frac{h\nu}{e^{\frac{h\nu}{k T}} - 1},
 \end{align}
 with $c$ is the speed of light in vacuum, $h$  the Planck constant and $k$
 the Boltzmann constant.

The emissivity vector defined above is a differential quantity that describes
emission from a volume element along an infinitesimal step along the propagation
path. This means that in order to obtain the emission from a finite volume, the
emission must be integrated along the propagation path. For a material that is
opaque, there is no need to integrate over the full volume, since only its
surface will contribute to the observed emission. The emissivity of a surface
can be described using an emissivity vector $\vec{e}$ in the same way as for
emission from a volume, with the difference that its components are unit less
and integration over the propagation path is not required.

\subsubsection{Absorption}

Absorption refers to the process of radiation being converted into internal
energy of the matter it interacts with. Mathematically, this process is
described by the absorption vector $\vec{\alpha}$, defined as the fraction of
the incoming radiation that is absorbed along an infinitesimal distance $ds$
along the propagation path:
\begin{align}
\vec{I}_\text{absorbed} &= (\vec{\alpha} \cdot\ ds) \odot \vec{I}
\end{align}
Here $\odot$ denotes the element-wise product of the absorption vector and
the Stokes vector $\vec{I}$ of the incoming radiation. Absorption may be
understood as the inverse process of thermal emission. Formally, this is
expressed by Kirhoff's  law of radiation
\begin{align}
  \vec{\alpha} &= \vec{\epsilon},
\end{align}
which states that the absorption vector is identical to the emissivity
vector defined in Eq.~\ref{eq:emissivity}.
This law is applicable to all matter in the atmosphere given that it is in a
state of local thermal equilibrium (LTE). LTE occurs when the density of matter
is sufficiently high so that the population rates of energy states above the
ground state are determined by thermal collisions rather than the absorption of
radiation. This decouples the emission of radiation from the radiation field,
allowing the simplified treatment of matter as thermal emitters with the
emission rates independent of the radiation field. LTE is a valid assumption for
radiative transfer in the troposphere.

\subsubsection{Scattering}

When a beam of radiation impinges upon a particle, their interaction may cause a
deviation of parts of the beam from the original propagation path. To first
order, scattering decreases the intensity of the beam. This particular process
is referred to as single scattering. As it propagates through the atmosphere,
the intensity of a beam is decreased by the effects of absorption and single
scattering. The combination of these two processes is referred to as attenuation
or extinction. As the rate of scattering increases, also the effect radiation
that is being scattered into the beam has to be taken into account.

Mathematically, the scattering of a beam of light propagating in direction
$\vec{n}$ into the direction $\vec{\hat{n}}$ is described by the phase
matrix $\mat{Z}(\vec{\hat{n}}, \vec{n})$:
\begin{align}
  \vec{I}_\text{scattered}(\vec{\hat{n}}) &= \mat{Z}(\vec{\hat{n}}, \vec{n}) \vec{I}(\vec{n})
\end{align}
The combined, attenuating effects of scattering and absorption are given by
the attenuation matrix $\mat{K}$, which is the sum of the absorption vector
$\vec{\alpha}$ and the fraction of radiation scattered away from the propagation
path:
\newcommand*{\vertbar}{\rule[-1ex]{0.5pt}{2.5ex}}
\newcommand*{\horzbar}{\rule[.5ex]{2.5ex}{0.5pt}}
\begin{align}
  \vec{K} &=
  \left [ \begin{array}{cccc}
      \vertbar & \vertbar & \vertbar & \vertbar \\
      \vec{\alpha} & \vec{0} & \vec{0} & \vec{0} \\
      \vertbar & \vertbar & \vertbar & \vertbar
    \end{array} \right ]
       + \int_{\vec{\hat{n}}} d\vec{\hat{n}}\ \mat{Z}(\vec{\hat{n}}, \vec{n})
\end{align}


\subsection{The radiative transfer equation}

The previous section introduced the fundamental interactions of radiation
with matter and how they are described mathematically in radiative transfer
theory. Combining the three processes of emission, absorption and scattering,
the change that a beam undergoes as it travels a distance $ds$ along its
propagation path through the atmosphere is described the vector radiative
transfer equation (VRTE):
\begin{align}\label{eq:vrte}
  \frac{d\vec{I}(\vec{n})}{ds} &=
  -\mat{K}\vec{I}(\vec{n}) + \vec{\alpha} \cdot B_\nu(T) + \int_{\hat{\vec{n}}} d\hat{\vec{n}} \ \mathbf{Z}(\vec{n}, \vec{\hat{n}}) \vec{I}(\vec{\hat{n}}).
  \end{align}

The first term on the right hand side is the extinction term, which represents
the combined effects of absorption and scattering of radiation out of the
propagation direction, and thus acts to decrease the intensity of the radiation
along the line of sight. The second term represents emission along the line of
propagation, with the emissivity vector $\bm{\epsilon}$ replaced by the
absorption vector $\bm{a}$ according to Kirchoff's law of thermal radiation. The
third term represents the radiation that is scattered into the line of sight.
Both of these terms act to increase the intensity along the line of sight.

\section{Application to the remote sensing of hydrometeors}

This section applies the principles of radiative transfer theory tha to
observations of hydrometeors in the atmosphere. The processes through which
hydrometeors affect satellite observations form the physical basis for the
remote sensing of clouds and precipitation. The discussion here is limited to
observations from the visible ($\lambda > \SI{380}{\nano \meter}$) to the
microwave domain ($\lambda > \SI{1}{\meter}$), which are the frequencies
that are most commonly used for the remote sensing of hydrometeors.

Most observations of clouds are affected not only by the clouds themselves but
also by other constituents of the atmosphere. In the following, we will
therefore first discuss observations without clouds, so called \textit{clear
  sky} observations, as these form the background for the observations of
hydrometeors. This is followed by a discussion of the observable effects of
hydrometeors on the observations, which gives rise to the signal in the
\textit{all-sky observations} that can be used to infer the physical properties
of hydrometeors.

\subsection{The Earth's surface}

The characteristics of the Earth's surface vary considerably across the
electromagnetic spectrum. In the visible range, the surface absorbs large parts
of the radation. The darkest surfaces are the ocean which absorbs around
$\SI{95}{\percent}$ of the incoming radiation. Bare land surface and forests are
considerable brighter but still absorb most ($\approx \SI{75}{\percent}$) of the
incoming radiation. In contrast to that, snow and ice covered surface reflect
nearly all of the incoming radiation and thus appear very bright.

In the infrared region of the electromagnetic spectrum the emissivity of the Earth surface
increases significantly. For wavelengths between $3$ and $\SI{20}{\micro \meter}$ the emissivity
of most surface types is larger than $0.9$. At these wavelengths the surface is very effective at
emitting and absorbing radiation and there is little contrast between different surface types.

In the microwave region surface emissivity patterns are more complex. Emissivity
from land is relatively high, while emissivity from water surfaces is low. The
contrast between land and surface increases with the wavelength. Emissivities
from snow and ice are lower than that of most bare land surface but higher than
that of water. The emissivity furthermore depends on the viewing angle and the
polarization of the radiation.

While these general tendencies are helpful for a qualitative analysis of
satellite imagery, it should be noted that they only provide a rough
characterization of the behavior of the Earth's surface across the
electromagnetic spectrum. Accurate, quantitative modeling of surface
emissivities is still an unsolved problem and thus remains an area of
active research.


\subsection{Clear-sky observations}

Satellites observe the Earth at a wide range of different frequencies. Since the
interactions of the various atmospheric constituents vary across the
electromagnetic spectrum, observations at different frequencies can provide
complementary information on the state of the atmosphere.

An important difference that separates observations at short wavelengths from
those at long wavelength is the source of the observed radiation. At wavelengths
in the visible and near infrared regions, the largest part of the observed
radiation stems from the sun. In the thermal infrared region and the even longer
wavelengths of the microwave region the observed radiation stems from the Earth's
surface or the atmosphere itself.

Gases affect radiation mostly through absorption and emission, while
scattering plays a role only in the visible range. The reason for this is
the small size of the molecules (around $\SI{1}{\nano \meter}$) compared
to the wavelengths of the radiation.

There is little absorption in the visible range with most of it due to water
vapor. In the near and thermal infrared absorption increases but is concentrated
in discrete absorption bands, which makes the opacity of atmosphere highly
variable across wavelengths. The principal gaseous absorbers in the infrared
region are water vapor, cabon dioxide and ozone. Absorption also plays an
important role in the microwave region with significant contributions from water
vapor, oxygen and ozone. 

Aerosols are significantly larger than the gas molecules in the atmosphere.
Because of that they affect both VIS and IR observations through scattering. In
the shortwave part of the electromagnetic spectrum, which is dominated by solar
radiation, aerosols reflect incoming solar radiation and which typically
increases the intensity of the measured radiation due to the relative darkness
of the surface. In the longwave infrared region, the scattering of aerosols acts
to decrease the intensity of the upwelling radiation from the surface and the
lower parts of the atmosphere. Aerosols have no significant effect on microwave
observations.

\subsection{Cloudy sky}

Hydrometeors generally interact with radiation through both scattering and
absorption. According to the radiative transfer equation in Eq.~\ref{eq:rte},
clouds thus have two competing effects on observed radiation. At each point
along the line of sight, the extinction term, which combines the effects of
absorption and scattering out of the line of sight, acts as a sink of radiation
moving towards the sensor. At the same time, emission and scattering into the
line of sight act as source term that increases the observed radiation.

In the visible and near infrared there is little absorption from either water
or ice. Hydrometeors thus mostly scatter the incoming solar radiation, which
causes them to appear bright. In the longwave IR both liquid water and ice
are highly absorptive. Because of this they absorb the upwelling radiation from
the surface and water vapor below. The radiation observed from the cloud
is therefore close to that of a blackbody of the ambient temperature at which the cloud
is located. These obesrvations therefore provide information on the altitude
of the clouds.

At microwave frequencies liquid water is a significantly more efficient absorber
than ice. Furthermore, the impact of scattering decreases with increasing
wavelength. At frequencies around $\sim \SI{50}{\giga \hertz}$ and below,
scattering is negligible and hydrometeors interact with radiation through
emission. At these low frequencies, thermal emission from liquid hydrometeors
over radiometrically cold Ocean surfaces causes a clear signal in the
observations. At frequencies above $\SI{50}{\giga \hertz}$, observations become
sensitive to the scattering from snow flakes. This scattering signal is
important for sensing precipitation over land, where the emission signature from
liquid hydrometeors may not be detectable due to the much warmer background
surface.


