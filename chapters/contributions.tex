
This chapter summarizes and discusses the research that is
presented in the appended articles.

\section{Handling retrieval uncertainty with neural networks}

The first study that was conducted as part of this thesis is the work presented
in \citet{pfreundschuh18} (Chap.~\ref{chap:qrnn}). It was motivated by the
incompatibility of available Bayesian retrieval methods with neural network
techniques. While the available Bayesian methods provided a principled way of
handling the uncertainties inherent to retrievals of atmospheric properties,
they were computationally expensive, which limited their applicability to
certain retrieval quantities, particularly those related to clouds and
precipitation. Retrievals based on neural networks often offered superior
performance both in terms of computational cost as well as accuracy but
typically neglected retrieval uncertainties.


The main result from this study is empirical evidence of the equivalence of
Bayesian retrieval methods and probabilistic neural network retrievals given
that the training data follows the a priori distribution of the Baysian
retrieval. The results are based on an idealized retrieval setup for which
samples cloud by generated from the true posterior distribution using Markov
Chain Monte Carlo (MCMC) sampling. For this case we were able to show that the
results from a QRNN-based retrieval are consistent with the results obtained
from MCMC. We furthermore found that the QRNN retrieval outperforms traditional
Bayesian methods on smaller datasets indicating that they are affected less by
the curse of dimensionality. Finally, we demonstrated the practicality of the
approach on a cloud-top pressure retrieval from infrared measurements that was
developed by EUMETSAT.

\section{Passive microwave precipitation retrievals}

The second study, presented in \citet{pfreundschuh22}
(Chap.~\ref{chap:gprof_nn}), follows up the methodological development from
\citet{pfreundschuh18} by applying the methodology to the retrieval of surface
precipitation from the GPM constellation. The study presents two neural networks
based implementations of the Goddard Profiling Algorithm (GPROF). The GPROF-NN
1D algorithm employs a fully-connected neural network to retrieve from a single
PMW observation. This version of the retrieval uses exactly the same input as
the GPROF algorithm. GPROF-NN 3D, the second proposed retrieval, is based on a
convolutional neural network and retrieves precipitation across the full swath
of observations simultaneously. This allows the retrieval to leverage structural
information in the observations, that is not available to the other retrievals.

Both the GPROF-NN 1D and GPROF-NN 3D retrieval were developed to be functionally
equivalent to the currently operational GPROF algorithm so that they can
potentially replace it in an upcoming update. This required the development of a
training pipeline that can be applied to all sensors of the GPM constellation.
This proved challenging especially for the GPROF-NN 3D retrieval. The training
data for all sensors of the GPM constellation is based on combined
radar/radiometer observations from the GMI sensor. Since these observations are
only available at a $\SI{100}{\kilo \meter}$-wide swath at the center of the GMI
swath, a ways had to be found to extent these to the full GMI-swath and to remap
them to the viewing geometries of the other sensors. The current approach uses
an intermediate simulator network to extend the simulated observations to the
full swath of GMI, which are then remapped to the viewing geometries of other
sensors by interpolation. This solution should be a considered a heuristic that
was pursued mainly because extending the simulations that are routinely performed
for the generation of the GPROF training data would have been outside the
scope of this study.

The main results from this study are estimates of potential improvements in
retrieval accuracy that can be realized by upgrading GPROF to either the
GPROF-NN 1D or GPROF-NN 3D retrieval. To isolate the effect of the retrieval
method from the actual training data, the retrieval accuracy was assessed using
held-out scenes from the training data. We found that consistent improvements in
a range of considered metrics and across all retrieval targets can potentially
be achieved by replacing GPROF with the GPROF-NN 1D algorithm. The accuracy
can be improved further by switching to the GPROF-NN 3D retrieval, which yields
additional improvements similar in magnitude to those provided by the GPROF-NN 1D
algorithm of the GPROF baseline retrieval. Moreover, we found that the effective
resolution of the retrieval improves by at least $\SI{40}{\percent}$ with the
neural network based retrieval. The study included results from two case studies
of Hurricane overpasses from the GMI and MHS sensors, which provided limited
evidence that the retrieval improvements can be expected to carry over to real
observations.

While the results from \citet{pfreundschuh22} were promising, their significance
was limited due to the evaluation being mostly based on observations from the
same distribution as the training data. Since for GMI the training data
consisted of real observations, these results can be expected to carry over to
operational application of the retrieval. For the other sensors, however, this
is less evident because their training consists of simulated observations whose
distribution deviates from that of real observations. In addition to this, the
question remained open to what extent the retrieval results that are closer to
the training data actually improve precipitation estimates compared to
independent validation data.

The study presented \citet{pfreundschuh22c} addresses these outstanding
questions. To validate the newly developed implementation of GPROF, the results
from the conventional and neural-network-based versions against ground-radar
measurements over the continental united states (CONUS) and the tropical
Pacific. The study was designed to provide a comprehensive assessment of the
errors of the GPROF retrievals in order to determine the improvements afforded
by neural-network based retrieval and to identify potential outstanding issues
impeding their adaptation to operational processing.

The validation against the ground based radar measurements largely confirmed the
results from \citet{pfreundschuh22}. Both neural-network-based retrievals
achieve significant improvements in retrieval accuracy compared to the
conventional implementation. This is also the for the effective resolution of
the retrievals, which are improved from $60$ to $\SI{20}{\kilo \meter}$ over
land surfaces and from $20$ to $\SI{15}{\kilo \meter}$ over Ocean.

\section{VIS/IR precipitation retrievals}

While the neural-network-based implementation of GPROF provided clear evidence
for the potential of neural-network based precipitation retrievals, the
constraint of a retrieval that provides the same output as the currently
operational GPROF algorithm limited the exploration of retrieval improvements to
currently available retrieval outputs. The assessments presented in
\citet{pfreundschuh22} and \citet{pfreundschuh22c} therefore focused mostly on
deterministic precipitation estimates and thus did not explore the full
potential of the probabilistic predictions afforded by the neural network.

The aim of the study presented in \citet{pfreundschuh22b} was to explore the
full potential of probabilistic neural-network based precipitation retrievals in
the context of near real-time retrievals from VIS/IR observations over Brazil.
The input data for the retrieval comes from the advanced baseline imager (ABI)
on the geostationary operational environmental satellite (GOES) 16. To train the
retrieval, the input observation were co-located with combined radar-radiometer
measurements from the GPM core observatory satellite.

To validate the retrieval its results were compared to 1 month of gauge
measurements. The retrieval accuracy was assessed by comparing it to the
retrieval that is currently used operationally at the Brazilian institute of
space research (INPE) as well as to commonly used global precipitation
retrievals. We were able to show that, despite the limited information content
of VIS/IR observations, deep-learning-based retrievals outperform currently
available methods, even those that merge IR observations from geostationary
satellite with PMW observations polar-orbiting platforms.

Furthermore, the study explored the potential of the probabilistic precipitation
estimates. We were able to show that, after correcting for differences in the
precipitation statistics of the training data and gauge measurements, confidence
intervals derived from the probabilistic predictions were well calibrated
against the gauge measurements. We also provided results that the probabilistic
predictions improve the detection of strong precipitation.


\section{Cloud correction for data assimilation}

The final study included in this thesis explores another retrieval application.
Motivated by an upcoming experimental satellite project, it addresses the
question how neural networks can be used to remove the cloud contamination from
microwave observations. This application is of interest as microwave
observations are an important source of information for weather forecasts.
However, the handling of observations that are affected by clouds is
computationally difficult and therefore not performed at all centers that use
these observations to generate forecasts.

A simple approach to improve the utilization of the available microwave
observations is to employ a correction scheme to remove the effect of the
clouds on the observations so that they can be ingested in the same way as
clear-sky observations. Since the correction effect cannot be
perfect, the cloud correction leads to increased uncertainties in the
corrected observations. Since the uncertainties assigned to observations play
an important role for their correct assimilation, a cloud correction scheme
should ideally be able to quantify the uncertainties in the corrected
observations.

In the study presented in \citet{kaur20}, we investigate the potential of QRNNs
to correct cloud-contaminated microwave observations using different channels at
different frequencies. The principal results from this study is that channels
around $\SI{325}{\giga \hertz}$ are better suited for cloud correction than
channels at $\SI{200}{\giga \hertz}$. In addition to that, the results show the
ability of QRNNs to accurately correct the microwave observations and predict
the associated uncertainties.




\section{Limitations and future work}
\begin{itemize}
  \item Discuss limitations of proposed approaches
  \item Discuss reamaining questions and new perspectives
\end{itemize}

