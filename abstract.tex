\thispagestyle{plain}
\begin{center}
  {\large \textbf{\thetitle}}
  
  \vspace{0.4cm}
  
  \vspace{0.4cm}
  {\sc \theauthor}
  
  \vspace{0.9cm}
  \textbf{Abstract}
\end{center}

Clouds, rain and snow are essential components of the global hydrological cycle
and the sole sustainable source of fresh water over land. They influence the
climate by modulating and distributing incoming solar energy. Clouds are
harbingers of storms, which may bring damaging amounts of precipitation leading
to floods or landslides. Because of their importance for both climate and
weather, significant scientific activity is devoted to improving the
understanding of their physics and how they are represented in weather and
climate models.

Satellite observations play an important role in this because of their ability
to provide global observations at regular time intervals. However, inferring
properties of clouds or precipitation from satellite observations is a
non-trivial task. Due to the limited information content of the observations and
the complex physics of the atmosphere, such \textit{retrievals} are affected by
considerable uncertainties. Traditional methods trade-off processing speed
against accuracy and the ability to characterize the uncertainties in their
predictions.

This thesis proposes and evaluates methods to perform retrievals of hydrometeors
(clouds and precipitation) with reliable uncertainty estimates using deep neural
networks. The practicality and benefits of the approach is demonstrated using
two real-world retrieval applications of precipitation. In both cases, we find
that the application of deep neural networks considerably improves the accuracy
of the retrieval over the traditional methods. In addition to that, the proposed
approach provides reliable uncertainty estimates while still fulfilling latency
requirements for real-time processing. In addition to that, show potential
fitness of the method for cloud correction in numerical weather prediction.

The principal advantage of the proposed approach is that it combines the
expressive power of deep neural networks with the theoretically sound treatment
of uncertainties of Bayesian retrieval methods, which are commonly used in the
field. The approach affords considerable improvements in accuracy compared to
traditional Bayesian methods due to the ability of the neural networks to 
scale to large datasets. For retrievals that are already based on neural
networks, it provides a simple and efficient way to quantify the dominant source
of uncertainty in the retrieval. The approach therefore has the potential to
improve a wide range of retrieval applications in addition to those presented
in this thesis.





