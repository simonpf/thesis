\thispagestyle{plain}
%\begin{center}
  {\large \textbf{\thetitle}}
  
  \vspace{0.4cm}
  
  \vspace{0.4cm}
  {\sc \theauthor} \\
  Department of Space, Earth and Environment \\
  Chalmers University of Technology
  
  \vspace{0.5cm}
  {\sc Abstract}
  \vspace{0.5cm}

The ability to observe and measure clouds and precipitation is essential for
climate science, meteorology and an increasing range of societal and economical
activities. This importance is due the role of clouds and precipitation in the
hydrological cycle as well as the weather and climate of the Earth. Patterns of
cloudiness and precipitation interact across continental scales and are highly
variable in both space and time. Their study and monitoring therefore requires
observations with global coverage and high temporal resolution, which currently
can only be provided by satellite observations.

Inferring properties of clouds or precipitation from satellite observations is a
non-trivial task. Due to the limited information content of the observations and
the complex physics of the atmosphere, such \textit{retrievals} are endowed with
significant uncertainties. Traditional methods to perform these retrievals trade
off processing speed against accuracy and the ability to characterize the
uncertainties in their predictions.

This thesis develops and evaluates two neural-network based methods for
performing retrievals of hydrometeors, i.e. clouds and precipitation, that are
capable of providing accurate predictions of the retrieval uncertainty. The
practicality and benefits of the proposed methods are demonstrated using three
real-world retrieval applications of cloud properties and precipitation. The
demonstrated benefits of these methods over traditional retrieval methods
 led to the adoption of one of the algorithms for operational use at the
European Organisation for the Exploitation of Meteorological Satellites. The two
other algorithms are planned to be integrated into the operational processing at
the Brazilian National Institute for Space Research as well as the processing of
observations from the Global Precipitation Measurement, a joint satellite
mission by NASA and the Japanese Aerospace Exploration Agency.

%The proposed methods are limited to the quantification of the irreducible
%uncertainty of the retrieval and thus based on the assumption that sufficiently
%large and realistic training data is available. Evaluation of the probabilistic
%retrievals against independent measurements indicates that this is a valid
%approximation for remote sensing retrievals. Although the methods cannot
%represent correlations between output variables, we show that the predicted
%retrieval uncertainties improve the characterization of the observed quantities,
%which should benefit downstream applications.

The principal advantage of the proposed methods is their simplicity and
computational efficiency. A minor modification of the architecture and training
of conventional neural networks is sufficient to capture the dominant source of
uncertainty for remote sensing retrievals. As shown in this thesis, deep neural
networks can significantly improve the accuracy of satellite retrievals of
hydrometeors. With the proposed methods, the benefits of modern neural network
architectures can be combined with reliable uncertainty estimates, which are
required to fully characterize the observed hydrometeors.

\vfill
{\textbf{Keywords:} Remote sensing, machine learning, clouds, precipitation}



