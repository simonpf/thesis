\thispagestyle{plain}
\begin{center}
  {\large \textbf{\thetitle}}
  
  \vspace{0.4cm}
  
  \vspace{0.4cm}
  {\sc \theauthor}
  
  \vspace{0.9cm}
  \textbf{Abstract}
\end{center}

Satellite observations of clouds and precipitation are a crucial source of
information in science, meteorology as well as an increasing range of societal
and economical activities. This importance is due to their role in the
hydrological cycle as well as the weather and climate of the Earth. Patterns of
cloudiness and precipitation are highly variable in both space and time and
exhibit interactions across continental scales. Their study therefore requires
observations with global coverage and high temporal resolution, which currently
can only be provided by satellite observations.

 However, inferring properties of clouds or precipitation from satellite
observations is a non-trivial task. Due to the limited information content of
the observations and the complex physics of the atmosphere,
such \textit{retrievals} are affected by considerable uncertainties. Traditional
methods trade-off processing speed against accuracy and the ability to
characterize the uncertainties in their predictions.

This thesis develops and evaluates methods to perform retrievals of
hydrometeors, i.e. clouds and precipitation, with reliable uncertainty estimates
using deep neural networks. The practicality and benefits of the approach are
demonstrated using three real-world retrieval applications of cloud top pressure
and precipitation. Because of the demonstrated benefits in these application,
one is already in operational use at the European Organisation for the
Exploitation of Meteorological Satellites, while the two others are planned to
be used for operational processing for the Global Precipitation Measurement, a
joint mission by NASA and the Japanese Aerospace Exploration Agency, and the
Brazilian National Institute for Space Research, respectively.

%The proposed methods are limited to the quantification of the irreducible
%uncertainty of the retrieval and thus based on the assumption that sufficiently
%large and realistic training data is available. Evaluation of the probabilistic
%retrievals against independent measurements indicates that this is a valid
%approximation for remote sensing retrievals. Although the methods cannot
%represent correlations between output variables, we show that the predicted
%retrieval uncertainties improve the characterization of the observed quantities,
%which should benefit downstream applications.

The principal advantage of the proposed methods is their simplicity and
computational efficiency. They only require a minor modification to architecture
and training of conventional neural networks but capture the dominant source of
uncertainty for remote sensing retrievals. For retrievals based on conventional
Bayesian retrieval algorithms, the methods provide equivalent probabilistic
estimates while unlocking superior accuracy and processing speed afforded by
neural networks. As demonstrated by the examples presented in this thesis, the
proposed methods provides a highly effective way to improve a wide range of
remote sensing applications.




